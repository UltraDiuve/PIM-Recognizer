\part{Travaux subséquents}

    \chapter{Amélioration de la performance}
        Tester d'autres embeddings
        
        Essayer de stacker plusieurs modèles
        
        Voir si on peut pas faire marcher le scoring relatif, ça semblait bien.

    \chapter{Opérationnalisation de cette maquette}    
        \section{Client et sponsor métier}
        Estimation du ROI et identification d'un sponsor et d'un client.

        \section{Sélection du use case}
        Préalimentation ou appui au contrôle de données ?

        \section{Mise en place d'une organisation projet}
            \subsection{Identification des compétences nécessaires}
            Nécessite des compétences diverses : développement côté PIM, compétences infra, définition du niveau de criticité de cette fonctionnalité (pour définition du monitoring et des plans de reprise d'activité)
            \subsection{Choix d'un cadre méthodologique projet}
            Scrum, c'est ce qu'on connaît le mieux.
            \subsection{Développement côté PIM}
            Développement des fonctionnalités telles qu'elles ont été présentées dans la section sur le choix du use case.


            Tant la préalimentation, que l'aide au contrôle des données seraient faisable d'un point de vue technique.
            Il suffirait pour cela de publier un service, qui fonctionnerait de la manière suivante : 
            \begin{itemize}
                \item le PIM appelle le service, avec un message contenant l'uid du produit à contrôler ou préalimenter
                \item le serveur récupère du PIM les données nécessaires au contrôle ou à la préalimentation
                \item en retour, il renvoie au PIM soit l'état du contrôle (OK, erreur, avertissement, avec les précisions nécessaires), soit les données telles qu'elles doivent être préalimentées
                \item le PIM, sur la base de ce retour, affiche le résultat du contrôle ou bien alimente les données et les présente à l'utilisateur
            \end{itemize}

        \section{Industrialisation du code du modèle}
        Refactoring de certaines classes (e.g. : le requester, qui porte trop de responsabilités)
        Poursuite de l'écriture de tests unitaire pour avoir une couverture > 80\%.
        Mise en place d'un processus de déploiement continu.
        Rédaction de la documentation : revue des docstring et mise en place d'un build Sphinx

        \section{Monitoring de la performance du modèle}
        voir la manière dont on peut superviser le niveau de performance du modèle.
        Capture-t-on en direct le retour des utilisateurs dans le PIM ?

    \chapter{Extension des fonctionnalités offertes}
        \section{Prise en compte de nouveaux types de pièces jointes}
        Aller chercher également les étiquettes.

        \section{Utilisation d'outil d'OCR pour les pdf non structurés}
        Intégrer ce qui a déjà été fait autour des solutions cloud, Google ou Azure.

        \section{Mise en place d'outil de spatialisation des textes}
        Charge importante, mais pourrait être utilisé pour d'autres sujets

        \section{Construction d'outils d'extraction de données connexes à la composition}
        Détermination des allergènes sur la base du contenu des listes d'ingrédients.

        \section{\'{E}largissement aux données nutritionnelles}
        Si spatialisation faisable, tenter de récupérer des données nutritionnelles.

        \section{\'{E}valuation de la performances sur d'autres familles de produits}
        Construction d'une nouvelle ground truth sur des fiches techniques de PassionFroid et TerreAzur, et évaluation de la performance du modèle.
