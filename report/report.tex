\documentclass{report}

\usepackage[utf8]{inputenc}
\usepackage[T1]{fontenc}
\usepackage[french]{babel}

\usepackage{lmodern}

\usepackage{setspace}
\spacing{1.5}

\makeatletter
\@addtoreset{chapter}{part}
\makeatother  

\usepackage{url}

\title{Extraction de données relatives aux produits alimentaires à partir de documents non structurés}
\author{Pierre \textsc{massé}}
\date{Juin 2020}

\begin{document}

\maketitle

\large
\begin{abstract}
    {\em
    La gestion de l'information produit est devenu un enjeu de société majeur ces dernières années.
    Les scandales sanitaires récents ont déclenché une prise de conscience collective des consommateurs, en parallèle de réglementations de plus en plus contraignantes pour l'ensemble des acteurs de la filière\cite{incotext}\cite{incoexpl}.
    }
\end{abstract}
\normalsize

\tableofcontents

\part{Contexte métier}
    \chapter{Description du Groupe}
        \section{Le métier du Groupe Pomona}
        \section{Les deux niveaux de décentralisation}
        \section{Les branches}
            \subsection{Les branches RHD}
            \subsection{Les branches spécialistes}
            \subsection{L'étranger}
    \chapter{La gestion de l'information produit}
        \section{L'information produit}
        \section{Le processus associé}
        \section{Le PIM (Product Information Management)}
\part{Les données disponibles}
    \chapter{Le périmètre produit}
    \chapter{Les données utilisables}
        \section{Données structurées}
        \section{Données non structurées}
        \section{Pièces jointes}
            \subsection{Fiches techniques fournisseur}
            \subsection{\'{E}tiquettes produit}
            \subsection{Fiches logistiques fournisseur}
            \subsection{Fiches techniques et argumentaires Pomona}
        \section{Un mot sur la qualité des données}
        \section{Les données \og manuellement étiquetées \fg}
\part{Les objectifs de ce projet}
    \chapter{Les cas d'usage}
        \section{Objectifs : Qualité et productivité}
        \section{La préalimentation d'information}
        \section{Le contrôle à la saisie fournisseur}
        \section{L'aide aux vérifications Pomona}
        \section{Les contrôles en masse asynchrones}
    \chapter{Le type de données à récupérer}
        \section{La composition produit}
        \section{Les données nutritionnelles}
        \section{Les données logistiques}

\appendix
\part{Figures et tableaux}
    \listoftables
    \listoffigures
\part{Bibliographie}
    \bibliographystyle{plain}
    \bibliography{./biblio}
\part{Exemple de documents fournisseur}
    \chapter{Fiches techniques}
    \chapter{\'{E}tiquettes produit}
\part{Le code utilisé}
    \chapter{Extraction de données du PIM}
    \chapter{Conversion des pièces jointes en textes}
    \chapter{Identification des listes d'ingrédients}

\end{document}